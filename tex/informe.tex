
\documentclass[a4paper, 10pt, twoside]{article}

\usepackage[top=1in, bottom=1in, left=1in, right=1in]{geometry}
\usepackage[utf8]{inputenc}
\usepackage[spanish, es-ucroman, es-noquoting]{babel}
\usepackage{setspace}
\usepackage{fancyhdr}
\usepackage{lastpage}
\usepackage{amsmath}
\usepackage{amsfonts}
\usepackage{amsthm}
\usepackage{verbatim}
\usepackage{fancyvrb}
\usepackage{graphicx}
\usepackage{float}
\usepackage{enumitem} % Provee macro \setlist
\usepackage{tabularx}
\usepackage{multirow}
\usepackage{hyperref}
\usepackage{xspace}
\usepackage[toc, page]{appendix}

\usepackage{tikz}
\usetikzlibrary{shapes.geometric, arrows}


%%%%%%%%%% Constantes - Inicio %%%%%%%%%%
\newcommand{\titulo}{Trabajo Práctico}
\newcommand{\materia}{Teoría de Lenguajes}
\newcommand{\integrantes}{Delgado · Lovisolo · Petaccio}
\newcommand{\cuatrimestre}{Segundo Cuatrimestre de 2014}
%%%%%%%%%% Constantes - Fin %%%%%%%%%%


%%%%%%%%%% Configuración de Fancyhdr - Inicio %%%%%%%%%%
\pagestyle{fancy}
\thispagestyle{fancy}
\lhead{\titulo\ · \materia}
\rhead{\integrantes}
\renewcommand{\footrulewidth}{0.4pt}
\cfoot{\thepage /\pageref{LastPage}}

\fancypagestyle{caratula} {
   \fancyhf{}
   \cfoot{\thepage /\pageref{LastPage}}
   \renewcommand{\headrulewidth}{0pt}
   \renewcommand{\footrulewidth}{0pt}
}
%%%%%%%%%% Configuración de Fancyhdr - Fin %%%%%%%%%%


%%%%%%%%%% Miscelánea - Inicio %%%%%%%%%%
% Evita que el documento se estire verticalmente para ocupar el espacio vacío
% en cada página.
\raggedbottom

% Separación entre párrafos.
\setlength{\parskip}{0.5em}

% Separación entre elementos de listas.
\setlist{itemsep=0.5em}

% Asigna la traducción de la palabra 'Appendices'.
\renewcommand{\appendixtocname}{Apéndices}
\renewcommand{\appendixpagename}{Apéndices}
%%%%%%%%%% Miscelánea - Fin %%%%%%%%%%


\begin{document}


%%%%%%%%%%%%%%%%%%%%%%%%%%%%%%%%%%%%%%%%%%%%%%%%%%%%%%%%%%%%%%%%%%%%%%%%%%%%%%%
%% Carátula                                                                  %%
%%%%%%%%%%%%%%%%%%%%%%%%%%%%%%%%%%%%%%%%%%%%%%%%%%%%%%%%%%%%%%%%%%%%%%%%%%%%%%%


\thispagestyle{caratula}

\begin{center}

\includegraphics[height=2cm]{DC.png} 
\hfill
\includegraphics[height=2cm]{UBA.jpg} 

\vspace{2cm}

Departamento de Computación,\\
Facultad de Ciencias Exactas y Naturales,\\
Universidad de Buenos Aires

\vspace{4cm}

\begin{Huge}
\titulo
\end{Huge}

\vspace{0.5cm}

\begin{Large}
\materia
\end{Large}

\vspace{1cm}

\cuatrimestre

\vspace{4cm}

\begin{tabular}{|c|c|c|}
\hline
Apellido y Nombre & LU & E-mail\\
\hline
Delgado, Alejandro N.  & 601/11 & nahueldelgado@gmail.com\\
Lovisolo, Leandro      & 645/11 & leandro@leandro.me\\
Petaccio, Lautaro José & 443/11 & lausuper@gmail.com\\
\hline
\end{tabular}

\end{center}

\newpage


%%%%%%%%%%%%%%%%%%%%%%%%%%%%%%%%%%%%%%%%%%%%%%%%%%%%%%%%%%%%%%%%%%%%%%%%%%%%%%%
%% Introducción                                                              %%
%%%%%%%%%%%%%%%%%%%%%%%%%%%%%%%%%%%%%%%%%%%%%%%%%%%%%%%%%%%%%%%%%%%%%%%%%%%%%%%


\section{Introducción}
Presentaremos en este informe, un intérprete encargado de procesar el lenguaje provisto por la cátedra y generar los renders tridimensionales asociados a este lenguaje.

Utilizaremos la librería Ply, sobre la cuál definiremos la gramática para este lenguaje, para luego construir el lexer y el parser que completarán el funcionamiento del intérprete.


\section{Especificación de la gramática}

\begin{verbatim}
Rules -> Rule_Definition Rules
       | Empty

Rule_Definition -> Rule = Element
                 | Rule .= Element
                 | $ = Element

Element -> Primitive
         | Rule
         | Transform
         | Element_And
         | Element_Or
         | Element_Power
         | Element_Group
         | Element_Optional

Primitive -> Box
           | Ball
           | Underescore

Rule -> Regla

Transform -> Element : Transform_Name Arith_Expr

Transform_Name -> Rx
                | Ry
                | Rz
                | Sx
                | Sy
                | Sz
                | S
                | Tx
                | Ty
                | Tz
                | Cr
                | Cg
                | Cb
                | D

Element_And -> Element & Element

Element_Or -> Element | Element

Element_Power -> Element ^ Arith_Expr

Element_Group -> [ Element ]

Element_Optional -> < Element >

Arith_Expr -> Airth_Expr_Number
            | Airth_Expr_Uplus
            | Airth_Expr_Uminus
            | Airth_Expr_Parenthesis
            | Arith_Expr_Plus
            | Airth_Expr_Minus
            | Airth_Expr_Times
            | Airth_Expr_Divide

Arith_Expr_Number -> Number

Arith_Expr_Uplus -> + Arith_Expr

Arith_Expr_Uminus -> - Arith_Expr

Arith_Expr_Times -> Arith_Expr * Airth_Expr

Arith_Expr_Divide -> Airth_Expr / Airth_Expr

Empty

\end{verbatim}

La gramática presentada presenta ambigüedad y proyecciones que pueden ser simplificadas. Se creó de esta manera para simplificar la implementación del parser.

Se utilizó la librería Ply para asignar las reglas de precedencia y asociatividad, desambiguando la misma.

Las reglas, utilizando los tokens y el tipo de asociatividad, de menor precedencia a mayor, son las siguientes:

\begin{enumerate}
\item Left $\to$ PLUS , MINUS
\item Left $\to$ TIMEs, DIVIDE
\item Right $\to$ UPLUS, UMINUS
\item Right $\to$ AND
\item Right $\to$ OR
\item Right $\to$ POWER
\item Right $\to$ COLON
\end{enumerate}

\section{Ejemplificación de árboles de derivación}

\tikzstyle{word} = [rectangle, rounded corners, minimum width=1cm,text centered]
\tikzstyle{arrow} = [thick,>=stealth]
\begin{tikzpicture}[node distance=1.5cm]

%$ = box : s 1 ^1

%Nivel uno
\node (pesosUno) [word, xshift=0.5cm] {\$};
\node (igualUno) [word, right of=pesosUno, xshift=0.5cm] {=};
\node (boxUno) [word, right of=igualUno, xshift=0.5cm] {box};
\node (puntosUno) [word, right of=boxUno, xshift=0.5cm] {:};
\node (escalaUno) [word, right of=puntosUno, xshift=0.5cm] {s};
\node (numeroUno) [word, right of=escalaUno, xshift=0.5cm] {1};
\node (powerUno) [word, right of=numeroUno, xshift=0.5cm] {pow};
\node (otroNumeroUno) [word, right of=powerUno, xshift=0.5cm] {1};

%Nivel dos
\node (primitiveBoxDos) [word, below of=boxUno] {Primitive};
\node (arithExprDos) [word, below of=numeroUno] {Arith\_Expr\_Number};

%nNivel tres

\node (elementTres) [word, below of=primitiveBoxDos] {Element};
\node (puntosTres) [word, right of=elementTres, xshift=0.5cm] {:};
\node (escalaTres) [word, right of=puntosTres, xshift=0.5cm] {Transform\_Name};


\draw [arrow] (boxUno) -- (primitiveBoxDos);
\draw [arrow] (numeroUno) -- (arithExprDos);
\draw [arrow] (primitiveBoxDos) -- (elementTres);
\draw [arrow] (puntosUno) -- (puntosTres);
\draw [arrow] (escalaUno) -- (escalaTres);

\end{tikzpicture}

\section{Modo de uso}

El programa admite como input, archivos con texto plano con las instrucciones para realizar el renderizado según el lenguaje.

Puede ejecutarse en la carpeta /src mediante el comando \verb+python2 main.py input_file.peg+

\section{Requerimientos}
Se requiere la utilización de Python 2.7 para la ejecución del programa.

Son necesarias además, las siguientes bibliotecas:
\begin{itemize}
\item Panda3D para el renderizado gráfico, descargable de https://www.panda3d.org/
\item Ply para el lado del parsing, descargable de http://www.dabeaz.com/ply/
\end{itemize}

El código del programa se encuentra en /src y debe ser ejecutado desde main.py

\section{Código}

\subsection{Main}

\begin{small}
\verbatiminput{../src/main.py}
\end{small}


\subsection{Lexer}

\begin{small}
\verbatiminput{../src/lexer.py}
\end{small}

\subsection{Parser}

\begin{small}
\verbatiminput{../src/parser.py}
\end{small}

\subsection{Scene}

\begin{small}
\verbatiminput{../src/scene.py}
\end{small}


\end{document}
