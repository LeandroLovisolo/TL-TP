
\documentclass[a4paper, 10pt, twoside]{article}

\usepackage[top=1in, bottom=1in, left=1in, right=1in]{geometry}
\usepackage[utf8]{inputenc}
\usepackage[spanish, es-ucroman, es-noquoting]{babel}
\usepackage{setspace}
\usepackage{fancyhdr}
\usepackage{lastpage}
\usepackage{amsmath}
\usepackage{amsfonts}
\usepackage{amsthm}
\usepackage{verbatim}
\usepackage{fancyvrb}
\usepackage{graphicx}
\usepackage{float}
\usepackage{enumitem} % Provee macro \setlist
\usepackage{tabularx}
\usepackage{multirow}
\usepackage{hyperref}
\usepackage{xspace}
\usepackage{qtree}
\usepackage[toc, page]{appendix}


%%%%%%%%%% Constantes - Inicio %%%%%%%%%%
\newcommand{\titulo}{Trabajo Práctico}
\newcommand{\materia}{Teoría de Lenguajes}
\newcommand{\integrantes}{Delgado · Lovisolo · Petaccio}
\newcommand{\cuatrimestre}{Segundo Cuatrimestre de 2014}
%%%%%%%%%% Constantes - Fin %%%%%%%%%%


%%%%%%%%%% Configuración de Fancyhdr - Inicio %%%%%%%%%%
\pagestyle{fancy}
\thispagestyle{fancy}
\lhead{\titulo\ · \materia}
\rhead{\integrantes}
\renewcommand{\footrulewidth}{0.4pt}
\cfoot{\thepage /\pageref{LastPage}}

\fancypagestyle{caratula} {
   \fancyhf{}
   \cfoot{\thepage /\pageref{LastPage}}
   \renewcommand{\headrulewidth}{0pt}
   \renewcommand{\footrulewidth}{0pt}
}
%%%%%%%%%% Configuración de Fancyhdr - Fin %%%%%%%%%%


%%%%%%%%%% Miscelánea - Inicio %%%%%%%%%%
% Evita que el documento se estire verticalmente para ocupar el espacio vacío
% en cada página.
\raggedbottom

% Separación entre párrafos.
\setlength{\parskip}{0.5em}

% Separación entre elementos de listas.
\setlist{itemsep=0.5em}

% Asigna la traducción de la palabra 'Appendices'.
\renewcommand{\appendixtocname}{Apéndices}
\renewcommand{\appendixpagename}{Apéndices}

\newcommand{\grafico}[3]{
  \begin{figure}[H]
  	\centering
    \includegraphics[height=10cm]{#1}
    \caption{#2}
    \label{#3}
  \end{figure}
}


%%%%%%%%%% Miscelánea - Fin %%%%%%%%%%


\begin{document}


%%%%%%%%%%%%%%%%%%%%%%%%%%%%%%%%%%%%%%%%%%%%%%%%%%%%%%%%%%%%%%%%%%%%%%%%%%%%%%%
%% Carátula                                                                  %%
%%%%%%%%%%%%%%%%%%%%%%%%%%%%%%%%%%%%%%%%%%%%%%%%%%%%%%%%%%%%%%%%%%%%%%%%%%%%%%%


\thispagestyle{caratula}

\begin{center}

\includegraphics[height=2cm]{DC.png} 
\hfill
\includegraphics[height=2cm]{UBA.jpg} 

\vspace{2cm}

Departamento de Computación,\\
Facultad de Ciencias Exactas y Naturales,\\
Universidad de Buenos Aires

\vspace{4cm}

\begin{Huge}
\titulo
\end{Huge}

\vspace{0.5cm}

\begin{Large}
\materia
\end{Large}

\vspace{1cm}

\cuatrimestre

\vspace{4cm}

\begin{tabular}{|c|c|c|}
\hline
Apellido y Nombre & LU & E-mail\\
\hline
Delgado, Alejandro N.  & 601/11 & nahueldelgado@gmail.com\\
Lovisolo, Leandro      & 645/11 & leandro@leandro.me\\
Petaccio, Lautaro José & 443/11 & lausuper@gmail.com\\
\hline
\end{tabular}

\end{center}

\newpage


%%%%%%%%%%%%%%%%%%%%%%%%%%%%%%%%%%%%%%%%%%%%%%%%%%%%%%%%%%%%%%%%%%%%%%%%%%%%%%%
%% Introducción                                                              %%
%%%%%%%%%%%%%%%%%%%%%%%%%%%%%%%%%%%%%%%%%%%%%%%%%%%%%%%%%%%%%%%%%%%%%%%%%%%%%%%


\section{Introducción}

En este trabajo desarrollamos un visualizador para el formato de síntesis procedural PEGS, creado por el cuerpo docente de la materia Teoría de Lenguajes de nuestra facultad.


\section{Especificación de la gramática}

A continuación detallamos primero los tokens reconocidos por el lexer y luego las producciones reconocidas por el parser.


\subsection{Tokens}

Las cadenas representadas por cada token se corresponden con su nombre en minúscula salvo especificado lo contrario.


\vspace{1em}
\begin{tabular}{ll}
\hline
Token & Descripción\\
\hline\hline
\texttt{BOX}, \texttt{BALL}                       & Primitivas\\
\texttt{UNDERSCORE}                               & Primitiva vacía (carácter ``\_'')\\
\texttt{CYLINDER}, \texttt{CONE}, \texttt{TORUS}  & Primitivas adicionales\\
\texttt{RX}, \texttt{RY}, \texttt{RZ}             & Transformaciones de rotación\\
\texttt{SX}, \texttt{SY}, \texttt{SZ}, \texttt{S} & Transformaciones de escala\\
\texttt{TX}, \texttt{TY}, \texttt{TZ}             & Transformaciones de traslación\\
\texttt{CR}, \texttt{CG}, \texttt{CB}             & Transformaciones de coloreo\\
\texttt{D}                                        & Transformación de límite de profundidad de recursión\\
\texttt{COLON}                                    & Inicio de una transformaciín (carácter ``:'')\\
\texttt{AND}, \texttt{OR}                         & Operaciones de conjunción y disyunción (caracteres ``\&'' y ``\textbar'')\\
\texttt{POWER}                                    & Operación de potenciación (carácter ``\^{}'')\\
\texttt{LGROUP}, \texttt{RGROUP}                  & Agrupación de operaciones (caracteres ``['' y ``]'')\\
\texttt{LOPT}, \texttt{ROPT}                      & Operación opcional (caracteres ``\textless'' y ``\textgreater'')\\
\texttt{NUMBER}                                   & Constante numérica (expresión regular \texttt{[0-9]+(\textbackslash.[0-9]+)?})\\
\texttt{PLUS}, \texttt{MINUS},
\texttt{TIMES}, \texttt{DIVIDE}                   & Operadores aritméticos binarios (caracteres ``+'', ``-'', ``*'' y ``/'')\\
\texttt{LPAREN}, \texttt{RPAREN}                  & Agrupación de operaciones aritméticas (caracteres ``('' y ``)'')\\
\texttt{START\_RULE}                               & Regla inicial (carácter ``\$'')\\
\texttt{RULE}                                     & Regla (expresión regular \texttt{[a-zA-Z]+})\\
\texttt{DOT}                                      & Regla final (carácter ``.'')\\
\texttt{EQUALS}                                   & Declaración de regla (carácter ``='')\\
\texttt{COMMENT}                                  & Comentario (expresión regular \texttt{"([\^{}\textbackslash\textbackslash\textbackslash{}n]|(\textbackslash\textbackslash.))*"})\\
\hline
\end{tabular}


\subsection{Producciones}

Las producciones a continuación están escritas en el formato consumido por la librería PLY.

La producción inicial de esta gramática es \texttt{rules}.

Notar que la producción \texttt{empty} no tiene cuerpo. Esto es porque, como su nombre señala, se corresponde con la producción vacía, comúnmente escrita $A \rightarrow \epsilon$ o $A \rightarrow \lambda$ en los libros de texto, y la librería PLY no provee ninguna sintaxis especial para tales producciones.

Notar además que algunas producciones utilizan una sintaxis especial \texttt{\%proc}. Esta sintaxis se explica en la próxima sección.

\begin{verbatim}
  rules : rule_definition rules
        | empty

  rule_definition : RULE       EQUALS element
                  | RULE DOT   EQUALS element
                  | START_RULE EQUALS element

  element : primitive
          | rule
          | transform
          | element_and
          | element_or
          | element_power
          | element_group
          | element_optional

  primitive : BOX
            | BALL
            | UNDERSCORE
            | CYLINDER
            | CONE
            | TORUS

  rule : RULE

  transform : element COLON transform_name arith_expr

  transform_name : RX
                 | RY
                 | RZ
                 | SX
                 | SY
                 | SZ
                 | S
                 | TX
                 | TY
                 | TZ
                 | CR
                 | CG
                 | CB
                 | D

  element_and : element AND element

  element_or : element OR element

  element_power : element POWER arith_expr

  element_group : LGROUP element RGROUP

  element_optional : LOPT element ROPT

  arith_expr : arith_expr_number
             | arith_expr_uplus
             | arith_expr_uminus
             | arith_expr_parenthesis
             | arith_expr_plus
             | arith_expr_minus
             | arith_expr_times
             | arith_expr_divide

  arith_expr_number : NUMBER

  arith_expr_uplus : PLUS arith_expr %prec UPLUS

  arith_expr_uminus : MINUS arith_expr %prec UMINUS

  arith_expr_parenthesis : LPAREN arith_expr RPAREN

  arith_expr_plus : arith_expr PLUS arith_expr

  arith_expr_minus : arith_expr MINUS arith_expr

  arith_expr_times : arith_expr TIMES arith_expr

  arith_expr_divide : arith_expr DIVIDE arith_expr

  empty :
\end{verbatim}


\subsection{Órdenes de precedencia y asociatividades}

Las ambiguedades en las producciones anteriores se resuelven con los órdenes de precedencia y asociatividades en la tabla a continuación.

Notar que la tabla incluye dos tokens ficticios \texttt{UPLUS} y \texttt{UMINUS}. Éstos se corresponden con los operadores unarios ``+'' y ``-'', que definen el signo de una expresión aritmética. Notar que tienen un orden de precedencia superior al de sus contrapartes binarios.

Estos tokens ficticios se referencian desde las producciones \texttt{arith\_expr\_uplus} y \texttt{arith\_expr\_uminus}, que usan los tokens \texttt{PLUS} y \texttt{MINUS}, respectivamente, pero que en el contexto de esas producciones particulares requieren un orden de precedencia distinto. La librería PLY provee un mecanismo para cambiar el orden de precedencia de un token en una producción: escribiendo \texttt{\%prec \textless{}TOKEN\textgreater} al final de una producción, el orden de precedencia del token usado en esa producción se reemplaza por el orden de precedencia del token \texttt{\textless{}TOKEN\textgreater} en ese contexto.


\vspace{1em}
\begin{tabular}{llll}
  \hline
  Orden & Token           & Asociatividad & Comentarios\\
  \hline\hline
  1     & \texttt{PLUS}   & Izquierda     & Operadores aritméticos binarios\\
  1     & \texttt{MINUS}  & Izquierda     &\\
  2     & \texttt{TIMES}  & Izquierda     &\\
  2     & \texttt{DIVIDE} & Izquierda     &\\
  \hline
  3     & \texttt{UPLUS}  & Derecha       & Operadores aritméticos unarios\\
  3     & \texttt{UMINUS} & Derecha       &\\
  \hline
  4     & \texttt{AND}    & Izquierda     & Operadores binarios sobre elementos\\
  5     & \texttt{OR}     & Izquierda     &\\
  \hline
  6     & \texttt{POWER}  & Izquierda     & Operadores unarios sobre elementos\\
  7     & \texttt{COLON}  & Izquierda     &\\
  \hline
\end{tabular}


\section{Implementación}

La ejecución del programa se divide en dos fases: parsing de la entrada y rendering de la escena. Explicamos ambas a continuación.


\subsection{Parsing}

Durante esta fase se ejecuta el parser generado con la librería PLY y se arma una instancia de la clase \texttt{Scene}, que captura la semántica de la entrada del programa.

\texttt{Scene} tiene una variable de instancia \texttt{Scene::rules} que mantiene un arreglo de instancias de la clase \texttt{Rule} o alguna de sus subclases. Este arreglo tiene un item por cada regla definida en la entrada. Por ejemplo, la entrada a continuación genera un arreglo \texttt{Scene::rules} con 4 items:

\begin{verbatim}
a  = box  & a : tx 1
a  = ball & a : tx 1
a. = box : cg 0
$  = a : d 10
\end{verbatim}

Las reglas iniciales y finales corresponden a las subclases \texttt{StartRule} y \texttt{FinalRule}. El resto a la clase \texttt{Rule}.

Cada regla se modela mediante un árbol de parsing con exactamente la misma estructura reconocida por el parser. Cada nodo de este árbol se representa con una instancia de alguna subclase de \texttt{Node}, definida de la siguiente manera:

\begin{verbatim}
  class Node:
    def __init__(self, scene):
      self.scene = scene
      self.children = []
    ...
\end{verbatim}

Notar que la clase \texttt{Rule} es subclase de \texttt{Node}. En consecuencia, cada instancia de \texttt{Rule} en el arreglo \texttt{Scene::rules} es el nodo raíz del árbol de parsing de alguna regla definida en la entrada.

Existe aproximadamente una subclase de \texttt{Node} por cada producción en la gramática. Por ejemplo, las producciones \texttt{element\_and} y \texttt{element\_or} se corresponden con las subclases \texttt{And} y \texttt{Or}, respectivamente.

La librería PLY permite asociar acciones semánticas a cada producción. Por ejemplo, la producción \texttt{arith\_expr\_number} está asociada a la siguiente función:

\begin{verbatim}
  def p_arith_expr_number(self, t):
    'arith_expr_number : NUMBER'
    t[0] = scene.Number(self.scene, t[1])
\end{verbatim}

Esta función asocia al resultado de reconocer esta producción (\texttt{t[0]}) una instancia de la clase \texttt{Number} (subclase de \texttt{Node}) que recibe el valor numérico del token \texttt{NUMBER} (\texttt{t[1]}).

De forma similar, la producción \texttt{arith\_expr\_plus} está asociada a la siguiente función:

\begin{verbatim}
  def p_arith_expr_plus(self, t):
    'arith_expr_plus : arith_expr PLUS arith_expr'
    t[0] = scene.Plus(self.scene, t[1], t[3])
\end{verbatim}

Notar que el lado derecho de esta producción incluye nodos no terminales. En este caso, los contenidos de \texttt{t[1]} y \texttt{t[3]} serán instancias de subclases de \texttt{Node} creadas por el código asociado a producciones \texttt{arith\_expr}. La función termina asignándole a \texttt{t[0]} una instancia de \texttt{Plus}, subclase de \texttt{Node}, que tiene como hijos a las dos instancias de \texttt{Node} resultantes de los nodos no terminales.

El resto de los nodos de un árbol de parsing de una regla cuaqluiera se instancian de forma similar.

A continuación se muestra una entrada de ejemplo \texttt{eg09.peg}:

\begin{verbatim}
  $ = box:tx -1 & ball:tx 1
\end{verbatim}

Y su correspondiente árbol de parsing, obtenido pasándole el parámetro \texttt{-p} al comando \texttt{pegv}:

\begin{verbatim}
Rule "$"
  And
    TX
      Box
      UMinus
        Number "1.0"
    TX
      Ball
      Number "1.0"
\end{verbatim}


\subsection{Rendering}

El rendering de la escena se hace por medio de la librería Panda3D. Ésta mantiene su propio árbol de objetos visibles en una escena.

Cada objeto tiene atributos posición, rotación y escala, todos sobre los ejes X, Y y Z. A su vez, un objeto puede tener subobjetos. Todos los atributos de un objeto hijo son relativos a los del objeto padre. De esta forma, basta con modificar los atributos de un objeto padre para afectar a los objetos hijos de la forma esperada intuitivamente.

Todas las subclases de \texttt{Node} (instanciadas en la fase de parsing) implementan un método \texttt{render()} que devuelve un objeto de Panda3D resultante de realizar la acción correspondiente a ése nodo. Por ejemplo, el método \texttt{Box::render()} devuelve un objeto de Panda3D correspondiente a un cubo de lado 1 centrado en el origen.

Un objeto de Panda3D devuelto por el método \texttt{render()} de una subclase de \texttt{Node} siempre está centrado en el origen. Esto facilita la aplicación de transformaciones sucesivas sin tener que calcular manualmente matrices de transformaciones.

En el caso que una instancia de \texttt{Node} tenga nodos hijos, al invocarse su método \texttt{render()}, ésta a su vez puede invocar al método \texttt{render()} de sus nodos hijos y/o crea un nuevo objeto de Panda3D. Este nuevo objeto de Panda3D puede o no tener como hijos a los objetos de Panda3D obtenidos de invocar al método \texttt{render()} de los nodos hijos. Algunos ejemplos:

\begin{itemize}
  \item Una instancia de \texttt{RZ}, correspondiente a rotar su nodo hijo sobre el eje Z, al invocarse su método \texttt{render()}, primero obtiene un objeto de Panda3D resultante de invocar al método \texttt{render()} de su nodo hijo, luego modifica su atributo rotación sobre el eje Z y devuelve dicho objeto.

  \item Una instancia de \texttt{TX}, correspondiente a modificar la posición sobre el eje X de su nodo hijo, al invocarse su método \texttt{render()}, primero crea un nuevo objeto de Panda3D inicialmente vacío, luego le agrega como hijo al objeto de Panda3D obtenido de invocar al método \texttt{render()} de su nodo hijo, modifica la posición del objeto hijo sobre el eje X y devuelve el objeto padre.

  \item Una instancia de \texttt{And}, al invocarse su método \texttt{render()}, crea un nuevo objeto de Panda3D inicialmente vacío, al que le agrega como hijos los objetos de Panda3D resultantes de invocar al método \texttt{render()} de sus dos nodos hijos.
\end{itemize}

El resto de las subclases de \texttt{Node} actúan de forma similar.

Dos subclases de \texttt{Node} particulares que valen la pena aclarar son \texttt{RuleElement} y \texttt{Power}.

La subclase \texttt{RuleElement} modifica la instancia de \texttt{Scene} agregándole estado sobre la regla que se está renderizando actualmente y el nivel de recursión actual (ver el diccionario \texttt{Scene::perRuleDepths}.) Además, en el caso de una llamada recursiva, decide si llamar a una regla general o a una regla final de acuerdo a si se alcanzó el límite de recursión particular para esa regla o global.

La subclase \texttt{Power} logra el efecto deseado creando nuevas instancias fantasma de \texttt{Power} con menor potencia, aplicándoles a dichas instancias fantasma las transformaciones requeridas y renderizando el nuevo nodo fantasma.

Finalmente, la escena entera se renderiza invocando el método \texttt{Scene::do\_render()}, que simplemente obtiene la regla inicial ``\$'' e invoca su método \texttt{render()}.


\section{Ejemplos de árboles de derivación}

Entrada:

\begin{verbatim}
  $ = box : tx 1
\end{verbatim}

Árbol de derivación:

\hspace{-5cm}
\Tree [
  .rules
    [.rule\_definition
      {RULE\\(\$)}
      !\qsetw{-5cm}
      EQUALS
      !\qsetw{-5cm}
      [.element
        [.transform
          [.element
            [.primitive BOX ]
          ]
          COLON
          [.transform\_name TX ]
          [.arith\_expr
            [.arith\_expr\_number {NUMBER\\(1)} ]
          ]
        ]
      ]
    ]
    !\qsetw{5cm}
    [.rules empty ]
]

\vspace{1em}

Entrada:

\begin{verbatim}
  a = BALL
  $ = a : s 2
\end{verbatim}

Árbol de derivación:

\hspace{-5cm}
\Tree [
  .rules
    [.rule\_definition
      {RULE\\(a)}
      EQUALS
      [.element BALL ]
    ]
    [.rules
      [.rule\_definition
        {RULE\\(\$)}
        !\qsetw{-8cm}
        EQUALS
        !\qsetw{-8cm}
        [.element
          [.transform
            [.element
              {RULE\\(a)}
            ]
            !\qsetw{-20cm}
            COLON
            [.transform\_name S ]
            [.arith\_expr
              [.arith\_expr\_number {NUMBER\\(2)} ]
            ]
          ]
        ]
      ]
      !\qsetw{5cm}
      [.rules empty ]
    ]
]

\vspace{1em}

Entrada:

\begin{verbatim}
  $ = box | ball | _
\end{verbatim}

Árbol de derivación:

\hspace{-5cm}
\Tree [
  .rules
    [.rule\_definition
      {RULE\\(\$)}
      !\qsetw{-5cm}
      EQUALS
      !\qsetw{-5cm}
      [.element
        [.element\_or
          [.element
            [.element\_or
              [.element
                [.primitive BOX ]
              ]
              OR
              [.element
                [.primitive BALL ]
              ]
            ]
          ]
          OR
          [.element
            [.primitive UNDERSCORE ]
          ]
        ]
      ]
    ]
    !\qsetw{5cm}
    [.rules empty ]
]


\section{Ejemplos}

Presentamos a continuación algunos ejemplos que ilustran la salida del visualizador.


\subsection{Dick Butt}

(Ver \url{http://knowyourmeme.com/memes/dick-butt}.)

Entrada:

\verbatiminput{../ejemplos/dickbutt.peg}

Resultado:

\grafico{dickbutt.png}{dickbutt.peg}{dickbutt}


\subsection{Globo}

Entrada:

\verbatiminput{../ejemplos/eg23.peg}

Resultado:

\grafico{ejemplo23.png}{eg23.peg}{eg23}


\subsection{Programa inválido 1}

Entrada:

\begin{verbatim}
  $ = ball : ts 1
\end{verbatim}

Salida:

\begin{verbatim}
  Syntax error at line 1, column 12:
  $ = ball : ts 1
             ^
\end{verbatim}

El token \texttt{ts} es inválido. El parser detecta esto y reporta correctamente el error.


\subsection{Programa inválido 2}

Entrada:

\begin{verbatim}
  ejes = box:sy0.05:sz0.05:cg0:cb0:tx0.5
  $ = ojos : tx 2
\end{verbatim}

Salida:

\begin{verbatim}
  LookupError: Rule ojos not found.
\end{verbatim}

La entrada es correcta desde el punto de vista gramatical, sin embargo se hace referencia a una regla que no existe. El visualizador reporta dicho error y termina la ejecución.


\section{Requerimientos}

\begin{itemize}
  \item Python 2.7
  \item Librería PLY (\url{http://www.dabeaz.com/ply/})
  \item Librería Panda3D (\url{https://www.panda3d.org/})
\end{itemize}


\section{Modo de uso}

Ejecutar \texttt{./pegv <archivo>} en el directorio raíz, reemplazando \texttt{<archivo>} por la ruta a algún archivo en formato PEGS. Ejemplo: \texttt{./pegv ejemplos/eg01.peg}.

Alternativamente proveer el código a interpretar por la entrada estándar. Por ejemplo: \texttt{cat ejemplos/eg01.peg | ./pegv}

Opcionalmente puede imprimirse cada regla reconocida y su respectivo árbol de parsing con la opción \texttt{-p}, por ejemplo: \texttt{./pegv -p ejemplos/eg01.peg}.


\section{Código fuente}


\subsection{main.py}

\begin{small}
  \verbatiminput{../src/main.py}
\end{small}


\subsection{lexer.py}

\begin{small}
  \verbatiminput{../src/lexer.py}
\end{small}


\subsection{parser.py}

\begin{small}
  \verbatiminput{../src/parser.py}
\end{small}


\subsection{scene.py}

\begin{small}
  \verbatiminput{../src/scene.py}
\end{small}


\end{document}
