
\documentclass[a4paper, 10pt, twoside]{article}

\usepackage[top=1in, bottom=1in, left=1in, right=1in]{geometry}
\usepackage[utf8]{inputenc}
\usepackage[spanish, es-ucroman, es-noquoting]{babel}
\usepackage{setspace}
\usepackage{fancyhdr}
\usepackage{lastpage}
\usepackage{amsmath}
\usepackage{amsfonts}
\usepackage{amsthm}
\usepackage{verbatim}
\usepackage{fancyvrb}
\usepackage{graphicx}
\usepackage{float}
\usepackage{enumitem} % Provee macro \setlist
\usepackage{tabularx}
\usepackage{multirow}
\usepackage{hyperref}
\usepackage{xspace}
\usepackage{qtree}
\usepackage[toc, page]{appendix}


%%%%%%%%%% Constantes - Inicio %%%%%%%%%%
\newcommand{\titulo}{Trabajo Práctico}
\newcommand{\materia}{Teoría de Lenguajes}
\newcommand{\integrantes}{Delgado · Lovisolo · Petaccio}
\newcommand{\cuatrimestre}{Segundo Cuatrimestre de 2014}
%%%%%%%%%% Constantes - Fin %%%%%%%%%%


%%%%%%%%%% Configuración de Fancyhdr - Inicio %%%%%%%%%%
\pagestyle{fancy}
\thispagestyle{fancy}
\lhead{\titulo\ · \materia}
\rhead{\integrantes}
\renewcommand{\footrulewidth}{0.4pt}
\cfoot{\thepage /\pageref{LastPage}}

\fancypagestyle{caratula} {
   \fancyhf{}
   \cfoot{\thepage /\pageref{LastPage}}
   \renewcommand{\headrulewidth}{0pt}
   \renewcommand{\footrulewidth}{0pt}
}
%%%%%%%%%% Configuración de Fancyhdr - Fin %%%%%%%%%%


%%%%%%%%%% Miscelánea - Inicio %%%%%%%%%%
% Evita que el documento se estire verticalmente para ocupar el espacio vacío
% en cada página.
\raggedbottom

% Separación entre párrafos.
\setlength{\parskip}{0.5em}

% Separación entre elementos de listas.
\setlist{itemsep=0.5em}

% Asigna la traducción de la palabra 'Appendices'.
\renewcommand{\appendixtocname}{Apéndices}
\renewcommand{\appendixpagename}{Apéndices}

\newcommand{\grafico}[3]{
  \begin{figure}[H]
  	\centering
    \includegraphics[height=10cm]{#1}
    \caption{#2}
    \label{#3}
  \end{figure}
}


%%%%%%%%%% Miscelánea - Fin %%%%%%%%%%


\begin{document}


%%%%%%%%%%%%%%%%%%%%%%%%%%%%%%%%%%%%%%%%%%%%%%%%%%%%%%%%%%%%%%%%%%%%%%%%%%%%%%%
%% Carátula                                                                  %%
%%%%%%%%%%%%%%%%%%%%%%%%%%%%%%%%%%%%%%%%%%%%%%%%%%%%%%%%%%%%%%%%%%%%%%%%%%%%%%%


\thispagestyle{caratula}

\begin{center}

\includegraphics[height=2cm]{DC.png} 
\hfill
\includegraphics[height=2cm]{UBA.jpg} 

\vspace{2cm}

Departamento de Computación,\\
Facultad de Ciencias Exactas y Naturales,\\
Universidad de Buenos Aires

\vspace{4cm}

\begin{Huge}
\titulo
\end{Huge}

\vspace{0.5cm}

\begin{Large}
\materia
\end{Large}

\vspace{1cm}

\cuatrimestre

\vspace{4cm}

\begin{tabular}{|c|c|c|}
\hline
Apellido y Nombre & LU & E-mail\\
\hline
Delgado, Alejandro N.  & 601/11 & nahueldelgado@gmail.com\\
Lovisolo, Leandro      & 645/11 & leandro@leandro.me\\
Petaccio, Lautaro José & 443/11 & lausuper@gmail.com\\
\hline
\end{tabular}

\end{center}

\newpage


%%%%%%%%%%%%%%%%%%%%%%%%%%%%%%%%%%%%%%%%%%%%%%%%%%%%%%%%%%%%%%%%%%%%%%%%%%%%%%%
%% Introducción                                                              %%
%%%%%%%%%%%%%%%%%%%%%%%%%%%%%%%%%%%%%%%%%%%%%%%%%%%%%%%%%%%%%%%%%%%%%%%%%%%%%%%


\section{Introducción}
Presentaremos en este informe, un intérprete encargado de procesar el lenguaje provisto por la cátedra y generar los renders tridimensionales asociados a este lenguaje.

Utilizaremos la librería Ply, sobre la cuál definiremos la gramática para este lenguaje, para luego construir el lexer y el parser que completarán el funcionamiento del intérprete.


\section{Especificación de la gramática}

A continuación detallamos primero los tokens reconocidos por el lexer y luego las producciones reconocidas por el parser.


\subsection{Tokens}

Las cadenas representadas por cada token se corresponden con su nombre en minúscula salvo especificado lo contrario.


\vspace{1em}
\begin{tabular}{ll}
\hline
Token & Descripción\\
\hline
\texttt{BOX}, \texttt{BALL}                       & Primitivas\\
\texttt{UNDERSCORE}                               & Primitiva vacía (carácter ``\_'')\\
\texttt{CYLINDER}, \texttt{CONE}, \texttt{TORUS}  & Primitivas adicionales\\
\texttt{RX}, \texttt{RY}, \texttt{RZ}             & Transformaciones de rotación\\
\texttt{SX}, \texttt{SY}, \texttt{SZ}, \texttt{S} & Transformaciones de escala\\
\texttt{TX}, \texttt{TY}, \texttt{TZ}             & Transformaciones de traslación\\
\texttt{CR}, \texttt{CG}, \texttt{CB}             & Transformaciones de coloreo\\
\texttt{D}                                        & Transformación de límite de profundidad de recursión\\
\texttt{COLON}                                    & Inicio de una transformaciín (carácter ``:'')\\
\texttt{AND}, \texttt{OR}                         & Operaciones de conjunción y disyunción (caracteres ``\&'' y ``\textbar'')\\
\texttt{POWER}                                    & Operación de potenciación (carácter ``\^{}'')\\
\texttt{LGROUP}, \texttt{RGROUP}                  & Agrupación de operaciones (caracteres ``['' y ``]'')\\
\texttt{LOPT}, \texttt{ROPT}                      & Operación opcional (caracteres ``\textless'' y ``\textgreater'')\\
\texttt{NUMBER}                                   & Constante numérica (expresión regular \texttt{[0-9]+(\textbackslash.[0-9]+)?})\\
\texttt{PLUS}, \texttt{MINUS},
\texttt{TIMES}, \texttt{DIVIDE}                   & Operadores aritméticos binarios (caracteres ``+'', ``-'', ``*'' y ``/'')\\
\texttt{LPAREN}, \texttt{RPAREN}                  & Agrupación de operaciones aritméticas (caracteres ``('' y ``)'')\\
\texttt{START\_RULE}                               & Regla inicial (carácter ``\$'')\\
\texttt{RULE}                                     & Regla (expresión regular \texttt{[a-zA-Z]+})\\
\texttt{DOT}                                      & Regla final (carácter ``.'')\\
\texttt{EQUALS}                                   & Declaración de regla (carácter ``='')\\
\texttt{COMMENT}                                  & Comentario (expresión regular \texttt{"([\^{}\textbackslash\textbackslash\textbackslash{}n]|(\textbackslash\textbackslash.))*"})\\
\hline
\end{tabular}


\subsection{Producciones}

\begin{verbatim}
rules -> rule_definition rules
       | empty

rule_definition -> rule = element
                 | rule .= element
                 | $ = element

element -> primitive
         | rule
         | transform
         | element_and
         | element_or
         | element_power
         | element_group
         | element_optional

primitive -> BOX
           | BALL
           | UNDERSCORE

rule -> RULE

transform -> element : transform_name arith_expr

transform_name -> RX
                | RY
                | RZ
                | SX
                | SY
                | SZ
                | S
                | TX
                | TY
                | TZ
                | CR
                | CG
                | CB
                | D

element_and -> element & element

element_or -> element | element

element_power -> element ^ arith_expr

element_group -> [ element ]

element_optional -> < element >

arith_expr -> airth_expr_number
            | airth_expr_uplus
            | airth_expr_uminus
            | airth_expr_parenthesis
            | arith_expr_plus
            | airth_expr_minus
            | airth_expr_times
            | airth_expr_divide

arith_expr_number -> number

arith_expr_uplus -> + arith_expr

arith_expr_uminus -> - arith_expr

arith_expr_times -> arith_expr * airth_expr

arith_expr_divide -> airth_expr / airth_expr

empty

\end{verbatim}

Tokens

\begin{itemize}
\item UNDERSCORE (\_) $\to$ primitiva nula
\item COLON (:) $\to$ inicio de transformación
\item AND (\&), OR ($\vert$), POWER ($ ^\wedge $) $\to$ operaciones
\item LGROUP ([), RGROUP (]) $\to$ agrupación de operaciones
\item LOPT ($<$), ROPT ($>$) $\to$ operación opcional
\item NUMBER $\to$ constantes numéricas
\item PLUS (+), MIN (-), TIMES (*), DIVIDE (/) $\to$ operaciones aritméticas
\item LPAREN ((), RPAREN ()) $\to$ agrupación de operaciones aritméticas
\item START\_RULE (\$), RULE, EQUALS ($=$) $\to$ reglas
\item DOT $\to$ regla final
\item COMMENT $\to$ comentarios
\item BOX, BALL $\to$ primitivas
\item CYLINDER, CONE, TORUS $\to$ primitivas agregadas
\item RX, RY, RZ $\to$ rotación
\item SX, SY, RZ $\to$ traslación
\item CR, CG, CB $\to$ color
\item D $\to$ máxima profundidad
\end{itemize}

La gramática presentada presenta ambigüedad y proyecciones que pueden ser simplificadas. Se creó de esta manera para simplificar la implementación del parser.

Se utilizó la librería Ply para asignar las reglas de precedencia y asociatividad, desambiguando la misma.

Las reglas, utilizando los tokens y el tipo de asociatividad, de menor precedencia a mayor, son las siguientes:

\begin{enumerate}
\item Left $\to$ PLUS , MINUS
\item Left $\to$ TIMEs, DIVIDE
\item Right $\to$ UPLUS, UMINUS
\item Right $\to$ AND
\item Right $\to$ OR
\item Right $\to$ POWER
\item Right $\to$ COLON
\end{enumerate}

\section{Ejemplificación de árboles de derivación}


\textbf{Entrada:}

\texttt{\$ = box : tx 1}

Árbol de derivación:

\hspace{-5cm}
\Tree [
  .rules
    [.rule\_definition
      {RULE\\(\$)}
      !\qsetw{-5cm}
      EQUALS
      !\qsetw{-5cm}
      [.element
        [.transform
          [.element
            [.primitive BOX ]
          ]
          COLON
          [.transform\_name TX ]
          [.arith\_expr
            [.arith\_expr\_number {NUMBER\\(1)} ]
          ]
        ]
      ]
    ]
    !\qsetw{5cm}
    [.rules empty ]
]

\textbf{Entrada:}

\texttt{myrule = BALL}

\texttt{\$ = myrule : s 2}

Árbol de derivación:

\hspace{-5cm}
\Tree [
  .rules
    [.rule\_definition
      {RULE\\(myrule)}
      !\qsetw{-5cm}
      EQUALS
      [.element
        BALL
      ]
    ]
    [.rules
    	[.rule\_definition
    		{RULE\\(\$)}
    		!\qsetw{-5cm}
      		EQUALS
      		[.element
      			[.transform
      				[.element
      					!\qsetw{-5cm}
      					{RULE\\(myrule)}
      				]
      				!\qsetw{-5cm}
      				COLON
      				!\qsetw{-5cm}
      				[.transform\_name S ]
      				[.arith\_expr
            			[.arith\_expr\_number {NUMBER\\(2)} ]
          			]
      			]
      		]
    	]
    !\qsetw{5cm}
    [.rules empty ]
    ]
]

\textbf{Entrada:}

\texttt{\$ = box | ball | \$}

Árbol de derivación:

\hspace{-5cm}
\Tree [
  .rules
    [.rule\_definition
      {RULE\\(\$)}
      !\qsetw{-5cm}
      EQUALS
      [.element
        [.element\_or
          [.element
            [.primitive BOX ]
          ]
          OR
          [.element
          	[.element\_or
          		[.element
          			[.primitive BOX ]
          		]
          		OR
          		[.element
          			[.rule
          				{RULE\\(\$)}
          			]
          		]
          	]
          ]
        ]
      ]
    ]
    !\qsetw{5cm}
    [.rules empty ]
]

\section{Ejemplos de programas válidos e inválidos}

\subsection{Programas válidos}

\noindent
Entrada:

\begin{verbatim}
balloon = ball:cg0:cb0:sx0.9:sz0.9                  "el globo"
        & ball:cg0:cb0:s0.04:ty-0.52                "el nudo"
        & box:cg0:cb0:s0.03:rz45:rx45:ty-0.56       "el pico"
        & line:ty-0.56:d10                          "el hilo"

line = box:sx0.01:sy0.1:sz0.01:ty-0.05 & [ line:rz10:ty-0.1 | line:rz-10:ty-0.1 ]

$ = balloon

\end{verbatim}

\grafico{ejemplo23.png}{eg23.peg}{Ej1}


\subsection{Programas inválidos}

\noindent
Entrada:
\begin{verbatim}
$ = ball : tx 2 : ts 1 : tl 2 : tz 1
\end{verbatim}

\noindent
Salida:
\begin{verbatim}
Syntax error at line 1, column 19:
$ = ball : tx 2 : ts 1 : tl 2 : tz 1
                  ^
\end{verbatim}
Los tokens ts y tl son inválidos, el parser detecta que ts no es un token y muestra la posición del token erróneo.

\noindent
Entrada:
\begin{verbatim}
ejes = box:sy0.05:sz0.05:cg0:cb0:tx0.5
$ = ejjjjes : tx 2
\end{verbatim}

\noindent
Salida:
\begin{verbatim}
LookupError: Rule ejjjjes not found.
\end{verbatim}

La regla ejjjjes no produce un error gramatical, pero si ocurre un error a la hora de realizar el renderizado, la regla no se encuntra definida, entonces se levanta un error indicando el problema.


\section{Modo de uso}

El programa admite como input, archivos con texto plano con las instrucciones para realizar el renderizado según el lenguaje.

Puede ejecutarse en la carpeta /src mediante el comando \verb+python2 main.py input_file.peg+

\section{Requerimientos}
Se requiere la utilización de Python 2.7 para la ejecución del programa.

Son necesarias además, las siguientes bibliotecas:
\begin{itemize}
\item Panda3D para el renderizado gráfico, descargable de https://www.panda3d.org/
\item Ply para el lado del parsing, descargable de http://www.dabeaz.com/ply/
\end{itemize}

El código del programa se encuentra en /src y debe ser ejecutado desde main.py

\section{Código}

\subsection{Main}

\begin{small}
\verbatiminput{../src/main.py}
\end{small}


\subsection{Lexer}

\begin{small}
\verbatiminput{../src/lexer.py}
\end{small}

\subsection{Parser}

\begin{small}
\verbatiminput{../src/parser.py}
\end{small}

\subsection{Scene}

\begin{small}
\verbatiminput{../src/scene.py}
\end{small}


\end{document}
